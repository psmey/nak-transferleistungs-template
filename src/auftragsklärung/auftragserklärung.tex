\documentclass[12pt]{article}

% internal inputs
% packages
\usepackage{xcolor} % for extended colors functionality
\usepackage{parskip} % to deactivate new paragraph intendation
\usepackage{fancyhdr} % to create headers
\usepackage{graphicx} % for graphic usage
\usepackage{geometry} % for page margins
\usepackage[OT1]{fontenc} % enable european encoding and font
\usepackage{sectsty} % enable heading command formatting
\usepackage[english, ngerman]{babel} % enable syllable separation (the last language is the selected document langue)
\usepackage{colortbl} % to use colors in tables
\usepackage{tabularx} % for changing table sizes with an factor based approach
\usepackage{acro} % acronyms list, alphabetically sorted
\usepackage{pdfpages} % to import pdfs into the document
\usepackage{subcaption} % allow images arrays with subfigures
\usepackage{csquotes} % for consistent quoting
\usepackage[style=authoryear]{biblatex} % better bibliography
\usepackage{float} % allows for [H] option for figures, which actually enforces position in text
\usepackage[outputdir=../../build]{minted} % for display of code blocks

% special characters
% http://mirrors.ctan.org/info/symbols/comprehensive/symbols-a4.pdf
\usepackage{amsfonts} % allow for \checkmark
\usepackage{pifont} % allow for \ding{228}

% dev tools
\usepackage{blindtext}

% document parameters

% Colors (HEX)
\definecolor{naklightblue}{HTML}{3cd2ff}
\definecolor{nakblue}{HTML}{003a79}
\definecolor{nakdarkblue}{HTML}{0f192d}

% set graphics path
\graphicspath{{../../assets/images}}

% import bibliography
\addbibresource{../../config/bibliography.bib}

% header and footer
%% macros (need to be changed with \renewcommand)
\renewcommand{\headrulewidth}{0pt} % set thickness of a line under the header
\renewcommand{\headruleskip}{0em} % set the distance between the line and the header text, only visual, does not impact header hight
\renewcommand{\footrulewidth}{0pt} % set thickness of a line above the footer
\renewcommand{\footruleskip}{0em} % set the distance between the line and the footer text, only visual, does not impact footer hight

\setlength{\headsep}{1em}
\setlength{\headheight}{47.07607pt}

% page layout
\geometry{%
    a4paper,
    margin=2cm,
    top=2.5cm,
}

% font
\renewcommand*\familydefault{\sfdefault} % set base font of the document to sans serif


% heading formats
\newcommand{\headingStandardFormat}{\color{nakblue}\textbf{}}

\sectionfont{\headingStandardFormat}
\subsectionfont{\headingStandardFormat}
\subsubsectionfont{\headingStandardFormat}

% handle pagenumbering
%% start pagenumberin for frontmatter
\newcommand{\frontmatter}{%
    \newcounter{frontmatterPage}
    \pagenumbering{Roman} % change numbering style
}

%% start pagenumbering of mainmatter
\newcommand{\mainmatter}{%
    \setcounter{frontmatterPage}{\value{page}} % safe frontmatter counter value
    \pagenumbering{arabic} % change numbering style
}

%% reapply pagenumbering from frontmatter
\newcommand{\backmatter}{%
    \pagenumbering{Roman} % change numbering style
    \setcounter{page}{\value{frontmatterPage}} % set pagenumbering to frontmatter counter
}

% enable \renewcommand for heading values
%% Through this command a single source of truth is created for sections
%% where the heading needs to be written more than once like:
%%% \renewcommand{\headingValue}{Heading}
%%% \subsection*{\headingValue{}}
%%% \addcontentsline{toc}{subsection}{\headingValue{}}
%% "Heading" is only written once, which can mitigate easily overseen errors
\newcommand{\headingValue}{}

% easy import for appendix
\newcommand{\addAppendix}[4]{%
    % parameters:
    % #1: appendix heading
    % #2: filename
    % #3: include pdf parameters
    \subsection{#1} % Heading for appendix
    \label{#3}
    %\addcontentsline{toc}{subsection}{#1} % register appendix in toc
    \setcounter{frontmatterPage}{\value{page} + 1} % safe page value
    \includepdf[#4]{../assets/appendix/#2} % add pdf into main
    \backmatter{} % set correct page value, to not count appendix pages
}

\newcommand{\nakQuestionTabular}[2]{%
    \renewcommand{\arraystretch}{1.5} % change vertical padding of the table

    \begin{tabularx}
        {\textwidth} % specify table width
        {| X |} % collumn rules, X fills space to collumn width
        \hline
        \rowcolor{nakblue}
        \textcolor{white}{#1} \\
        \hline
        #2                    \\
        \hline
    \end{tabularx}

    \renewcommand{\arraystretch}{1} % change vertical padding of the table
}

% list symbol
\renewcommand{\labelitemi}{\color{nakblue}\ding{228}}

% minted

%% define highlighting style
\usemintedstyle{vs}

%% custom listingscaption
\renewcommand{\listingscaption}{Quelltext}

%% global minted environment style
\setminted{%
    autogobble,
    linenos, % enable line numbers
    % numbers = left, % default none
    % numbes is essentially the same as linenos exceot the side can be specified
    tabsize = 4 % default 8
}

%% custom line numbers
\renewcommand{\theFancyVerbLine}{%
    \textcolor{gray}{%
        \ttfamily
        \scriptsize
        \oldstylenums{\arabic{FancyVerbLine}}
    }
}

% hyperref
\hypersetup{%
    breaklinks=true, % default value: false
    colorlinks=true, % default value: false
    linkcolor=black, % default value: red
    % anchorcolor=black, % default value: black
    citecolor=black, % default value: green
    filecolor=black, % default value: cyan
    urlcolor=black, % default value: magenta
    bookmarksopen=true,
}

\DeclareAcronym{BA}{
    short = BA,
    long = Beispiel Abkürzung
}

\DeclareAcronym{ABA}{
    short = ABA,
    long = Andere Beispiel Abkürzung
}
\def\transferleistungsNr{[NR]}
\def\matrikelnummer{[MATRIKELNUMMER]}
\def\freigegebenesThema{[THEMA]}
\def\studiengangZenturie{[STUDIENGANG, ZENTURIE]}

\begin{document}

% header
\pagestyle{fancy} % enable fancyhdr package
\fancyhf{} % clear header and footer content
\fancyhead[C]{\includegraphics[width=\textwidth]{header.png}} % add header content
\fancyfoot[C]{\thepage}

\section*{Auftragsklärung Transferleistungen Theorie/Praxis}

\vspace{1.5em}

Matrikelnummer: \matrikelnummer{}

\vspace{1.5em}

Da Sie eine Fragestellung bzw.\ eine Problemlösung für Ihr Unternehmen anstreben, ist eine
umfangreiche Auftragsklärung unabdingbar und hat drei wesentliche Ziele:

\begin{itemize}
    \item \textbf{\color{nakblue} Sie} haben von Beginn an ein klares Bild, was die Anforderungen Ihres Unternehmens an Ihre Transferleistung sind,
    \item \textbf{\color{nakblue} die betrieblichen Betreuer} Ihrer Unternehmen wissen, welchen Output und somit Mehrwert sie von Ihnen durch die Transferleistung erwarten können,
    \item \textbf{\color{nakblue} die NORDAKADEMIE} kann überprüfen, ob Ihr gewähltes Thema in sich konsistent erscheint, realistisch vom Umfang ist und ob Sie dies mit gewählten Mitteln erfolgreich bearbeiten können.
\end{itemize}

\vspace{1.5em}

Bitte beantworten Sie die Fragen gemeinsam mit Ihrem betrieblichen Betreuer. Insgesamt stehen Ihnen als Richtwert 2.000 Zeichen zur Verfügung.

\vspace{1.5em}

\nakQuestionTabular{%
    Wie lautet Ihr Thema (die betriebliche Fragestellung/das betriebliche Problem)?%
}{%
    [ANTWORT]
}

\vspace{1.5em}

\nakQuestionTabular{%
    Was ist der Anlass für die Fragestellung bzw.\ das Problem? \newline
    Wie wurde ich oder mein betrieblicher Betreuer auf das Problem aufmerksam und was wurde ggf.\ schon unternommen?%
}{%
    [ANTWORT]
}

\vspace{1.5em}

\nakQuestionTabular{%
    Wie stellt sich die Fragestellung bzw.\ das Problem konkret in der Arbeitssituation dar? \newline
    Was bedeutet es, die Fragestellung bzw.\ das Problem noch nicht beantwortet bzw.\ gelöst zu haben?%
}{%
    [ANTWORT]
}

\vspace{1.5em}

\nakQuestionTabular{%
    Was ist das Ziel der Transferleistung?%
}{%
    [ANTWORT]
}

\vspace{1.5em}

\nakQuestionTabular{%
    Wie ist das Vorgehen bei der Beantwortung der Fragestellung/des Problems geplant? \newline
    Mit welchen Methoden/Materialien will ich dies herausfinden/belegen/prüfen? \newline
    Was will ich in meiner Transferleistung hauptsächlich tun: argumentieren? analysieren? vergleichen? interpretieren? prüfen?%
}{%
    [ANTWORT]
}

\newpage

\vspace*{1.5em}

Ich gebe das Thema hiermit frei und übernehme die Betreuung des Studierenden während
der Erstellung der Transferleistung:

\vspace{1.5em}

\renewcommand{\arraystretch}{1.5} % change vertical padding of the table

\begin{tabularx}{.4\textwidth}{X}
    \\
    \hline
    Unterschrift betriebl. Betreuer \\
\end{tabularx}

\vspace{3em}

Im Anschluss laden Sie die
Auftragsklärung im CIS hoch. Ob Ihre Fragestellung angenommen wurde, erfahren Sie innerhalb einer Woche (gerechnet an Werktagen). Wenn das Thema durch die NORDAKADEMIE angenommen wurde, startet der 4-wöchige
Bearbeitungszeitraum. Wird Ihre Transferleistung nicht angenommen, erhalten Sie eine
Information über den Ablehnungsgrund und können die Fragestellung, z. B. mit einem
anderen Schwerpunkt oder einem anderen geplanten Vorgehen, erneut hochladen.
Dieser Prozess kann sich ggf.\ mehrmals wiederholen, bis die Fragestellung angenommen
wurde.

\end{document}
