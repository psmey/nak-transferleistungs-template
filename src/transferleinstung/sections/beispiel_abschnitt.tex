\section{Beispiel Abschnitt}
\parencite{latex_companion} ist eine Beispielreferenz.
Es können Abkürzungen verwendet werden, wie \ac{BA} aber auch \ac{ABA}.

\subsection{Beispiel Unterabschnitt}

\begin{figure}[h]
    \centering

    \begin{subfigure}{.4\textwidth}
        \centering
        \includegraphics[width=.8\linewidth]{NAK_logo_full_text}
        \caption{Logo 1}
        \label{fig:subfig1}
    \end{subfigure}
    \begin{subfigure}{.4\textwidth}
        \centering
        \includegraphics[width=.8\linewidth]{NAK_logo_full_text}
        \caption{Logo 2}
        \label{fig:subfig2}
    \end{subfigure}

    \caption{Nordakademie Logos}
    \label{fig:fig}
\end{figure}

In Abb.~\ref{fig:subfig1} ist ein Beispiel Bild zu sehen.

\begin{listing}[H]
    \begin{minted}[
        numbers = left
    ]{c}
int main() {
    // print hello world
    printf("hello, world");
    return 0;
}
    \end{minted}
    \caption{Hello world in C}
    \label{lst:listing}
\end{listing}

\subsubsection{Beispiel Unterunterabschnitt}

\Blindtext{}
